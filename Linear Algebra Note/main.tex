% pdflatex
\documentclass[nofonts,colorlinks]{tufte-handout}
\usepackage{amsmath,amsfonts,amsthm,amssymb}
\usepackage{bm}
\usepackage{pgf,tikz}
\usepackage{enumerate}
\usepackage[inline]{enumitem}
\usepackage{sectsty}

\usepackage[osf]{mathpazo} % Palatino for main and math
\usepackage[scaled=1.08]{newtxtext}
\usepackage[scaled=0.90]{helvet} % Helvetica for sans serif
\usepackage[scaled=0.85]{beramono} % Bera Mono for monospaced

% \usepackage{fontspec}
% \usepackage{unicode-math}
% \setmainfont{TeX Gyre Termes}
% \setmathfont{TeX Gyre Pagella Math}


% ---------- Preamble ----------
\pagestyle{fancyplain} % Make all pages in the document conform to the custom headers and footers
\fancyhead{} % No page header
\fancyfoot[L]{}
\fancyfoot[C]{}
\fancyfoot[R]{\thepage}
\renewcommand{\headrulewidth}{0pt} % Remove header underlines
\renewcommand{\footrulewidth}{0pt} % Remove footer underlines
\setlength{\headheight}{13.6pt} % Customize the height of the header
\allowdisplaybreaks
\geometry{
	left=13mm,
	textwidth=130mm,
	marginparsep=8mm,
	marginparwidth=55mm,
}
\setlist[enumerate]{
	label=(\alph*),
	leftmargin=2.5em,
	topsep=0.5em,
	itemsep=0em,
}
\allsectionsfont{\normalfont\bfseries}
\setcounter{secnumdepth}{-1}


% ---------- Theorem Environments ----------
\theoremstyle{plain} % default
\newtheorem{thm}{Theorem}
\newtheorem{thms}[thm]{Theorems}
\newtheorem{cor}[thm]{Corollary}
\newtheorem{prop}[thm]{Proposition}
\newtheorem{props}[thm]{Propositions}
\newtheorem{lem}[thm]{Lemma}
\newtheorem{conj}[thm]{Conjecture}
\newtheorem{quest}[thm]{Question}

\theoremstyle{definition}
\newtheorem{defn}[thm]{Definition}
\newtheorem{defns}[thm]{Definitions}
\newtheorem{exmp}[thm]{Example}
\newtheorem{exmps}[thm]{Examples}
\newtheorem{notn}[thm]{Notation}
\newtheorem{notns}[thm]{Notations}
\newtheorem{exer}[thm]{Exercise}

\theoremstyle{remark}
\newtheorem{rmk}[thm]{Remark}
\newtheorem{rmks}[thm]{Remarks}

\def\idea{\textit{\color[rgb]{0,0,.55}Proof Idea. }}

\newcounter{peq} % Set a new counter for problem-equations
\counterwithin{peq}{thm}
\newenvironment{peq}{%
   \equation
   \refstepcounter{peq}
%    \tag{P.~\thepeq}
   \tag{\thepeq}
}{%
   \endequation
}


% ---------- Symbol Macros ----------
\newcommand{\bra}[1]{\mathopen{}\left(#1\right)}
\newcommand{\sbra}[1]{\mathopen{}\left[#1\right]}
\newcommand{\cbra}[1]{\mathopen{}\left\{#1\right\}}
\newcommand{\norm}[1]{\mathopen{}\left\lVert#1\right\rVert}
\newcommand{\inp}[2]{\mathopen{}\left\langle#1,#2\right\rangle}
\newcommand{\abs}[1]{\mathopen{}\left|#1\right|}
\newcommand{\rest}[2]{\mathopen{}\left.#1\right|_{#2}}

\renewcommand{\epsilon}{\varepsilon}
\renewcommand{\phi}{\varphi}
\newcommand{\dnei}{\overset{\circ}{U}}

\newcommand{\R}{\mathbf{R}}
\newcommand{\N}{\mathbf{N}}
\newcommand{\Z}{\mathbf{Z}}
\newcommand{\C}{\mathbf{C}}
\newcommand{\Q}{\mathbf{Q}}
\newcommand{\F}{\mathbf{F}}
\renewcommand{\L}{\mathcal{L}}
\newcommand{\M}{\mathcal{M}}
\renewcommand{\P}{\mathcal{P}}
\newcommand{\B}{\mathcal{B}}
\newcommand{\E}{\mathcal{E}}
\newcommand{\zero}{\mathbf{0}}

\renewcommand{\intercal}{\mathrm{t}}
\renewcommand{\d}{\mathrm{d}}
\newcommand{\e}{\mathrm{e}}

\DeclareMathOperator{\llim}{\underset{\nti}{\underline{\lim}}}
\DeclareMathOperator{\ulim}{\underset{\nti}{\overline{\lim}}}
\def \nti {\mathnormal{n}\to\infty}
\def \tseries {{\textstyle\sum\limits_{n=1}^{\infty}}\,} % textstyle series
\def \dseries {\sum_{n=1}^{\infty}\,} % displaystyle series
\def \tprod {{\textstyle\prod\limits_{n=1}^{\infty}}\,} % textstyle prod
\def \dprod {\prod_{n=1}^{\infty}\,} % displaystyle prod
\def \vs {\vspace{-1em}}
\DeclareMathOperator{\dist}{dist}
\DeclareMathOperator{\spn}{span}
\DeclareMathOperator{\im}{im}
% \DeclareMathOperator{\ker}{ker} % already existed
\DeclareMathOperator{\tr}{tr}
\DeclareMathOperator{\rank}{rank}


% ---------- Title Section ----------
\title{	
	\normalfont\normalsize 
	{\itshape Linear Algebra Done Right} \\ [0pt]
	\huge Notes -- Linear Algebra
}
\author{Zhijie Chen}
\date{\vspace{-5pt}\normalsize\today}


\begin{document}
\justifying
\maketitle
\tableofcontents
\newpage

\thispagestyle{empty}
\begin{fullwidth}
	\Large
	\setstretch{1.4}
	\vspace*{\fill}
	\begin{center}
		Don't just read it; fight it!\\
		Ask your own questions.\\
		Look for your own examples.\\
		Discover your own proofs.\\
		Is the hypothesis necessary?\\
		Is the converse true?\\
		What happens in the classical special case?\\
		What about the degenerate cases?\\
		Where does the proof use the hypothesis?
	\end{center}
	\begin{flushright}
		\textemdash\,Paul Holmos\phantom{mmmmmmmmmmmmmm}
		% \textemdash\,Paul Holmos\phantom{mmmmmmmmmmm}
	\end{flushright}
	\vspace*{\fill}
\end{fullwidth}
\newpage


% \section{Vector Spaces}
% \begin{lem}[Linear dependence lemma]
% 	Suppose $v_1,\dots,v_m$ is a linearly dependent set in $V$. Then there exists $k\in\{1,2,\dots,m\}$ such that
% 	\[v_k\in\spn(v_1,\dots,v_m).\]
% 	Furthermore, removing the $k^\text{th}$ term from the list does not change the span.
% \end{lem}

% \begin{thm}
% 	Any two bases of a finite-dimensional vector space have the same length.%
% 	\footnote{This proposition ensures that the \emph{dimension} of a vector space is well-defined.}
% \end{thm}
% \begin{proof}
% 	Suppose $V$ is finite-dimensional. Let $\mathcal{B}_1$ and $\mathcal{B}_2$ be two bases of $V$. Considering $\mathcal{B}_1$ as an independent set and $\mathcal{B}_2$ as a spanning set leads to $\#\mathcal{B}_1\leq\#\mathcal{B}_2$. Interchanging the roles of $\mathcal{B}_1$ and $\mathcal{B}_2$ and we have $\#\mathcal{B}_2\leq\#\mathcal{B}_1$. Thus $\#\mathcal{B}_1=\#\mathcal{B}_2$.
% \end{proof}


\section{Linear Maps}
% \subsection{Kernels and Images}
% \begin{exer}
% 	Suppose $U$ and $V$ are finite-dimensional and $S\in\L\bra{V,W}$ and $T\in\L\bra{U,V}$. Prove that
% 	\[\dim\ker ST \leq \dim\ker S + \dim\ker T.\]
% \end{exer}
% \begin{proof}
% 	Restrict to $Z=\ker ST$. By the fundamental theorem of linear maps,
% 	\begin{align*}
% 		\dim Z & = \dim T(Z) + \dim\ker \rest{T}{Z} \\
% 		& \leq \dim T(Z) + \dim\ker T \\
% 		& = \dim ST(Z) + \dim\ker \rest{S}{T(Z)} + \dim\ker T \\
% 		& \leq \dim\ker S + \dim\ker T.\qedhere
% 	\end{align*}
% \end{proof}

% \begin{cor}[Sylvester's rank inequality]
% 	Suppose $A\in\F^{m,n}$ and $B\in\F^{n,p}$ are two matrices. Then%
% 	% \footnote{There is a slicker proof for this inequality using block matrices. But the proof here using linear maps is more informative.}
% 	\[\rank A+\rank B-n\leq\rank\bra{AB}.\]
% \end{cor}


\subsection{Products and Quotients of Vector Spaces}
% \begin{exer}
% 	Suppose $V_1,\dots,V_m$ are vector spaces. Prove that $\L\bra{V_1\times\cdots\times V_m,W}$ and $\L\bra{V_1,W}\times\cdots\times\L\bra{V_m,W}$ are isomorphic vector spaces. 
% \end{exer}
% \begin{proof}
% 	We construct an isomorphism $T$ between the two vector spaces. For every $\Gamma\in\L\bra{V_1\times\cdots\times V_m,W}$, define $\phi_k:V_k\to W$ for each $k$ by
% 	\[\phi_k\bra{v_k}=\Gamma\bra{0,\dots,v_k,\dots,0}\]
% 	with $v_k$ in the $k^\text{th}$ slot and $0$ in all other slots. It can be verified that $\phi_k\in\L\bra{V_k,W}$.

% 	Define $T$ by $T\bra{\Gamma}=\bra{\phi_1,\dots,\phi_m}$. It can be verified that $T$ is a linear map. We prove $T$ is an isomorphism by constructing its inverse linear map $S$.

% 	For every $\bra{\phi_1,\dots,\phi_m}\in\L\bra{V_1,W}\times\cdots\times\L\bra{V_m,W}$, let
% 	\[S\bra{\phi_1,\dots,\phi_m}\bra{v_1,\dots,v_m}=\phi_1\bra{v_1}+\cdots+\phi_m\bra{v_m}.\]
	
% 	It can be shown that $S$ is a linear map, and that $S\circ T=I$ and $T\circ S=I$. That proves $T$ is indeed an isomorphism between the two vector spaces.
% \end{proof}

\begin{prop}\label{prop: test for translate}
	A nonempty subset $A$ of $V$ is a translate of some subspace of $V$ if and only if $\lambda v+\bra{1-\lambda}w\in A$ for all $v,w\in A$ and all $\lambda\in\F$.
\end{prop}

% \begin{prop}\label{prop: direct sum from quotient space}
% 	Suppose $U$ is a subspace of $V$ and $v_1+U,\dots,v_m+U$ is a basis of $V/U$ and $u_1,\dots,u_n$ is a basis of $U$. Then $v_1,\dots,v_m,u_1,\dots,u_n$ is a basis of $V$. In other words, $V=\spn(v_1,\dots,v_m)\oplus U$.%
%     \footnote{$V=\spn(v_1,\dots,v_m)\oplus U$ still holds without the hypothesis that $U$ is finite-dimensional.}
% \end{prop}

\begin{exer}
	Suppose $U$ is a subspace of $V$ such that $V/U$ is finite-dimensional.
	\begin{enumerate}
		\item Prove that if $W$ is a finite-dimensional subspace of $V$ and $V=U+W$, then $\dim W\geq\dim V/U$.
		\item Prove that there exists a finite-dimensional subspace $W$ of $V$ such that $V=U\oplus W$ and $\dim W=\dim V/U$.
	\end{enumerate}
\end{exer}
\begin{proof}
	Let $\overline{w}_1+U,\dots,\overline{w}_m+U$ be a basis of $V/U$. It can be shown that $V=\spn(\overline{w}_1,\dots,\overline{w}_m)\oplus U$. Let $W_0=\spn(\overline{w}_1,\dots,\overline{w}_m)$. Then $V=U\oplus W_0$, as desired.

	Now we prove that for each subspace $W$ of $V$ such that $V=U+W$, we have $\dim W\geq m=\dim V/U$. For each $\overline{w}_k$ above, by definition we have $\overline{w}_l=u_k+w_k$ for some $u_k\in U$ and $w_k\in W$. It can be shown from the linear independence of $\overline{w}_1+U,\dots,\overline{w}_m+U$ that $\overline{w}_1-u_1,\dots,\overline{w}_m-u_m$ are independent vectors in $W$. Hence $\dim W\geq m$.
\end{proof}


\subsection{Duality}
\begin{prop}\label{prop: relation between subspace and annihilator}
	Suppose $V$ is finite-dimensional and $U$ is a subspace of $V$. Then%
	\footnote{Compare this result with $\bra{U^\perp}^\perp=U$ where $U$ is a finite-dimensional subspace of inner product space $V$.}
	\[U=\cbra{v\in V:\phi(v)=0\,\text{ for every }\,\phi\in U^0}.\]
\end{prop}

\begin{exer}
	Suppose $V$ is finite-dimensional and $\phi_1,\dots,\phi_n$ is a basis of $V'$. Prove that there exists a basis of $V$ whose dual basis is precisely $\phi_1,\dots,\phi_n$.
\end{exer}
\begin{proof}
	We start from an arbitrary basis $u_1,\dots,u_n$ of $V$. Let $\psi_1,\dots,\psi_n$ be its dual basis. In this proof, we take the standard basis $e_1,\dots,e_n$ as the basis of $\F^n$.
	
	Define $S,T\in\L\bra{V,\F^n}$ by
	\[T(v)=\bra{\phi_1(v),\dots,\phi_n(v)},\quad S(v)=\bra{\psi_1(v),\dots,\psi_n(v)}.\]
	Then $\M\bra{S,\bra{u_1,\dots,u_n}}=I$.
	
	Let $A$ be the change of basis matrix from $\psi$'s to $\phi$'s, i.e.,
	\[\begin{pmatrix}\phi_1&\cdots&\phi_n\end{pmatrix}=\begin{pmatrix}\psi_1&\cdots&\psi_n\end{pmatrix}A.\]
	Then by the definition of change of basis matrix, we have
	\begin{align*}
		\M\bra{T,\bra{u_1,\dots,u_n}}&=\begin{pmatrix}
			\M\bra{\phi_1,\bra{u_1,\dots,u_n}}\\
			\vdots\\
			\M\bra{\phi_n,\bra{u_1,\dots,u_n}}\\
		\end{pmatrix}
		=A^\intercal\begin{pmatrix}
			\M\bra{\psi_1,\bra{u_1,\dots,u_n}}\\
			\vdots\\
			\M\bra{\psi_n,\bra{u_1,\dots,u_n}}\\
		\end{pmatrix}\\\\
		&=A^\intercal\cdot\M\bra{S,\bra{u_1,\dots,u_n}}=A^\intercal.
	\end{align*}

	Consider basis $v_1,\dots,v_n$ of $V$ such that the change of basis matrix from $u$'s to $v$'s is $\bra{A^\intercal}^{-1}$.%
    \footnote{\idea The change of basis for $V'\to V'$ corresponds to the transpose of $V\leftarrow V$, where transpose and inverse both come from duality.}
	Thus
	\[\M\bra{T,\bra{v_1,\dots,v_n}}=\M\bra{T,\bra{u_1,\dots,u_n}}\cdot\M\bra{I,\bra{v_1,\dots,v_n},\bra{u_1,\dots,u_n}}=I.\]
	Then the dual basis of $v_1,\dots,v_n$ is precisely $\phi_1,\dots,\phi_n$, as desired.
\end{proof}

\begin{exer}[A natural isomorphism from primal space onto double dual space]
	Define $\Lambda:V\to V''$ by
	\[(\Lambda v)(\phi)=\phi(v)\]
	for each $v\in V$ and $\phi\in V'$.
	\begin{enumerate}
		\item Prove that if $T\in\L(V)$, then $T''\circ\Lambda=\Lambda\circ T$.
		\item Prove that if $V$ is finite-dimensional, then $\Lambda$ is an isomorphism from $V$ onto $V''$.
	\end{enumerate}
\end{exer}

\begin{rmks}
	\begin{enumerate}
		\item Suppose $V$ is finite-dimensional. Then $V$ and $V'$ are isomorphic, but finding an isomorphism from $V$ onto $V'$ generally requires choosing a basis of $V$. In contrast, the isomorphism $\Lambda$ from $V$ onto $V''$ does not require a choice of basis and thus is more natural.
		\item Another natural isomorphism is $\pi'\in\L\bra{(V/U)',V'}$ where $\pi$ is the normal quotient map.
	\end{enumerate}
\end{rmks}


\section{Polynomials}
\begin{thm}\label{thm: limited number of zeros of polynomials}
	Suppose $p\in\P(\F)$ is a nonconstant polynomial of degree $m$. Then $p$ has at most $m$ zeros in $\F$.%
	\footnote{A useful corollary of this theorem: when a polynomial $p$ has too many zeros, $p=0$.}
\end{thm}

\begin{rmk}
	Theorem~\ref{thm: limited number of zeros of polynomials} implies that the coefficients of a polynomial are uniquely determined. In particular, the \emph{degree} of a polynomial is well-defined.
\end{rmk}

\begin{thm}[Division algorithm for polynomials]
	Suppose that $p,s\in\P(\F)$, with $s\neq0$. Then there exist unique polynomials $q,r\in\P(\F)$ such that $p=sq+r$.%
\end{thm}
\begin{proof}
	Let $n=\deg p$ and $m=\deg s$. The case where $n<m$ is trivial. Thus we now assume that $n\geq m$.

	The list
	\[1,z,\dots,z^{m-1},s,zs,\dots,z^{n-m}s\]
	is linearly independent in $\P_n(\F)$. And it also has length $n+1$. Hence the list is a basis of $\P_n(\F)$.

	Because $p\in\P_n(\F)$, there exist unique constants $a_0,\dots,a_{m-1},b_0,\dots,b_{n-m}\in\F$ such that
	\begin{align*}
		p&=a_0+a_1z+\dots+a_{m-1}z^{m-1}+b_0s+b_1zs+\dots+b_{n-m}z^{n-m}s\\
		&=\bra{a_0+a_1z+\dots+a_{m-1}z^{m-1}}+s\bra{b_0+b_1z+\dots+b_{n-m}z^{n-m}}.\qedhere
	\end{align*}
\end{proof}

\begin{exer}
	Suppose $p,q\in\P(\C)$ are nonconstant polynomials with no zeros in common. Let $m=\deg p$ and $n=\deg q$. Prove that there exist $r\in\P_{n-1}(\C)$ and $s\in\P_{m-1}(\C)$ such that $rp+sq=1$.%
	\footnote{\idea Define $T:\P_{n-1}(\C)\times\P_{m-1}(\C)\to\P_{m+n-1}(\C)$ by $T(r,s)=rp+sq$ and prove the surjectivity of $T$.}
\end{exer}
% \begin{proof}
% 	Define $T:\P_{n-1}(\C)\times\P_{m-1}(\C)\to\P_{m+n-1}(\C)$ by $T(r,s)=rp+sq$. It can be shown that $T$ is an injective linear map. Because the domain space and target space have the same dimension, $T$ is surjective, completing the proof.
% \end{proof}


\section{Eigenvalues and Eigenvectors}
\subsection{Invariant Subspaces}
\begin{exer}
	Suppose that $\lambda_1,\dots,\lambda_n\in\R$ are pairwise distinct. Prove that the list $\e^{\lambda_1x},\dots,\e^{\lambda_nx}$ is linearly independent in $\R^{\R}$.
\end{exer}
\begin{proof}
	Let $V=\spn(\e^{\lambda_1x},\dots,\e^{\lambda_nx})$.%
	\footnote{Alternatively we can let $V=D(\R)$.}
	Define $D\in\L(V)$ by $Df=f'$. Then $\e^{\lambda x}$ is an eigenvector of $D$ corresponding to $\lambda$. A list of eigenvectors corresponding to distinct eigenvalues is linearly independent.
\end{proof}

\begin{exer}
	Suppose $V$ is finite-dimensional, $T\in\L(V)$, and $U$ is a subspace of $V$ invariant under $T$. Prove that each eigenvalue of the quotient operator $T/U$ is an eigenvalue of $T$.
\end{exer}
\begin{proof}
	It suffices to show that $T/U-\lambda I=(T-\lambda I)/U$ is not injective $\implies$ $T-\lambda I$ is not injective. We prove that $T-\lambda I$ is invertible $\implies$ $(T-\lambda I)/U$ is injective.

	Suppose $T-\lambda I$ is invertible. $U$ being invariant under $T$ implies that $U$ is invariant under $T-\lambda I$. Thus $(T-\lambda I)v\in U\iff v\in U$. Suppose $((T-\lambda I)/U)(v+U)=0$. Then $(T-\lambda I)v\in U$, which implies that $v\in U$, i.e., $v+U=0$. That proves the injectivity of $(T-\lambda I)/U$.
\end{proof}

\begin{exer}
	Suppose $V$ is finite-dimensional and $T\in\L(V)$. Prove that $T$ has an eigenvalue if and only if there exists a subspace of $V$ of dimension $\dim V-1$ that is invariant under $T$.%
	\footnote{\idea Consider the zero entries in $\M(T)$ and its transpose matrix $\M(T')$.}
\end{exer}
\begin{proof}
	We first suppose that $T$ has an eigenvalue $\lambda$. Then $\lambda$ is an eigenvalue of $T'$. There exists $\phi\in V'$ such that $\phi\circ T=T'\phi=\lambda\phi$. Extend $\phi$ to a basis $\phi,\phi_2,\dots,\phi_n$ of $V'$ and let $v,v_2,\dots,v_n$ be the basis of $V$ whose dual basis is $\phi,\phi_2,\dots,\phi_n$. Then $(\phi\circ T)v_k=\lambda\phi(v_k)=0$ for every $k$. Because $\phi(T v_k)=0$ for every $k$, we have $Tv_k\in\spn(v_2,\dots,v_n)$. That proves that $\spn(v_2,\dots,v_n)$ is invariant under $T$.
	
	Reversing the steps above leads to an eigenvector of $T'$, completing the proof.
\end{proof}


\subsection{The Minimal Polynomial}
\begin{exer}[Companion matrix of a polynomial]
	Suppose $a_0,\dots,a_{n-1}\in\F$. Let $T\in\L(\F^n)$ be such that $\M(T)$ (with respect to the standard basis) is
	\[\begin{pmatrix}
		0&&&&&-a_0\\
		1&0&&&&-a_1\\
		&1&0&&&-a_2\\
		&&\ddots&&&\vdots\\
		&&&&0&-a_{n-2}\\
		&&&&1&-a_{n-1}\\
	\end{pmatrix}\]
	Prove that the minimal polynomial of $T$ is the polynomial
	\[a_0+a_1z+\dots+a_{n-1}z^{n-1}+z^n.\]
\end{exer}

\begin{rmk}
	This exercise implies that every monic polynomial is the minimal polynomial of some operator. Hence an algorithm that could produce exact eigenvalues for each operators on each $\F^n$ does not exist.
\end{rmk}

\begin{exer}
	Prove that every operator on a finite-dimensional vector space of dimension at least $2$ has an invariant subspace of dimension $2$.
\end{exer}
\begin{proof}
	Let $T\in\L(V)$ and $n=\dim V$. We use induction on $n$. The base case $n=2$ is trivial. Now suppose $n>2$ and the desired result holds for all smaller positive integers. Let $p$ be the minimal polynomial of $T$.

	If $T$ has an eigenvalue $\lambda$, then $p(z)=q(z)(z-\lambda)$ for some monic polynomial $q$ with $\deg q=\deg p-1$. Because $\rest{q(T)}{\im(T-\lambda I)}=0$, the desired result holds by induction hypothesis if $\dim\im(T-\lambda I)\geq2$. If $T-\lambda I=0$ the desired result trivially holds. If $\dim\im(T-\lambda I)=1$, then $(T-\lambda I)v$ is a scalar multiple of some fixed $u\in V$ for all $v\in V$. Take $w\in V\backslash\spn(u)$ and $\spn(u,w)$ will satisfy the desired property.

	If $T$ has no eigenvalues, then $\F=\R$ and $p(z)=q(z)(z^2+bz+c)$ for some $b,c\in\R$ with $b^2<4c$ and monic polynomial $q$ with $\deg q=\deg p-2$. If $\dim\im(T^2+bT+cI)\geq2$ the desired result holds by induction hypothesis. If $T^2+bT+cI=0$, then let $w\in V$ be such that $w\neq0$. It can be verified that $\spn(w,Tw)$ is invariant under $T$. If $\dim\im(T^2+bT+cI)=1$, because $\dim\ker(T^2+bT+cI)$ is even, $n$ is odd, which implies that $T$ has an eigenvalue. That completes the proof.
\end{proof}



\subsection{Commuting Operators}
\begin{exer}
	Suppose $\E\subseteq\L(V)$ and every element of $\E$ is diagonalizable. Prove that there exists a basis of $V$ with respect to which every element of $\E$ has a diagonal matrix if and only if every pair of elements of $\E$ commutes.%
	\footnote{This exercise extends simultaneous diagonalizability to more than $2$ (possibly infinitely many) operators.}
\end{exer}
\begin{proof}
	Suppose every pair of elements of $\E$ commutes. We use induction on $n=\dim V$. The base case $n=1$ is trivial. Now suppose $n>1$ and the desired result holds for all smaller integers. Without loss of generality, suppose $\E\cap\{\lambda I:\lambda\in\F\}=\emptyset$, or else consider $\E\backslash\{\lambda I:\lambda\in\F\}$.

	Let $\lambda_1,\dots,\lambda_m$ be the distinct eigenvalues of $T\in\E$. Then $V=E(\lambda_1,T)\oplus\dots\oplus E(\lambda_m,T)$. Because $E(\lambda_k,T)$ is invariant under every $S\in\E$, it suffices to show that the desired result holds on $E(\lambda_1,T)$. Because $E(\lambda_1,T)\subsetneqq V$, it holds by induction hypothesis. The other direction is trivial.
\end{proof}

\begin{exer}
	Suppose $V$ is a finite-dimensional nonzero complex vector space. Suppose that $\E\subseteq\L(V)$ is such that $S$ and $T$ commute for all $S,T\in\E$.
	\begin{enumerate}
		\item Prove that there is a vector in $V$ that is an eigenvector for every element of $\E$.
		\item Prove that there exists a basis of $V$ with respect to which every element of $\E$ has an upper-triangular matrix.%
		\footnote{This exercise extends simultaneous upper triangularizability to more than $2$ (possibly infinitely many) operators.}
	\end{enumerate}
\end{exer}


\section{Inner Product Spaces}
\subsection{Inner Products and Norms}
\begin{thms}[Polarization identities]
	\begin{enumerate}
		\item Suppose $V$ is a real inner product space. Then
		\[\inp{u}{v}=\frac{\norm{u+v}^2-\norm{u-v}^2}{4}.\]
		\item Suppose $V$ is a complex inner product space. Then
		\[\inp{u}{v}=\frac{\norm{u+v}^2-\norm{u-v}^2+\norm{u+iv}^2i-\norm{u-iv}^2i}{4}.\]
	\end{enumerate}
	\label{thm: polarization identities}
\end{thms}

\begin{exer}\label{exer: condition for a norm-induced inner product}
	Prove that if $\norm{\cdot}$ is a norm on $U$ satisfying the parallelogram equality, then there is an inner product $\inp{\cdot}{\cdot}$ on $U$ such that $\norm{u}=\inp{u}{u}^{1/2}$ for all $u\in U$.%
	\footnote{Recall Theorems~\ref{thm: polarization identities}.}
\end{exer}

% \begin{exer}
% 	Suppose $f,g$ are differentiable functions from $\R$ to $\R^n$.
% 	\begin{enumerate}
% 		\item Prove that
% 		\[\inp{f(t)}{g(t)}'=\inp{f'(t)}{g(t)}+\inp{f(t)}{g'(t)}.\]
% 		\item Suppose $\norm{f(t)}\equiv c>0$ for every $t\in\R$. Prove that $\inp{f'(t)}{f(t)}=0$ for every $t\in\R$.%
% 		\footnote{This result has a nice geometric interpretation.}
% 	\end{enumerate}
% \end{exer}


\subsection{Orthonormal Bases}
\begin{exer}
	Suppose $v_1,\dots,v_m$ is a linearly independent list in $V$. Prove that the orthonormal list produced by the formulas of the Gram-Schmidt procedure is the only orthonormal list $e_1,\dots,e_m$ in $V$ such that $\inp{v_k}{e_k}>0$ and $\spn(v_1,\dots,v_k)=\spn(e_1,\dots,e_k)$ for each $k$.
\end{exer}

% \begin{exer}
% 	Suppose $V$ is finite-dimensional. Suppose $\inp{\cdot}{\cdot}_1,\inp{\cdot}{\cdot}_2$ are inner products on $V$ with corresponding norms $\norm{\cdot}_1$ and $\norm{\cdot}_2$. Prove that there exists $b,c>0$ such that $b\norm{v}_2\leq\norm{v}_1\leq c\norm{v}_2$ for every $v\in V$.%
% 	\footnote{Note that \textit{Proof 1} does not use the hypothesis that $\norm{\cdot}_1$ and $\norm{\cdot}_2$ are associated with an inner product respectively.}
% \end{exer}
% % \begin{proof}[Proof 1]
% % 	Let $u_1,\dots,u_n$ be a basis of $V$. Without loss of generality, suppose that%
% % 	\footnote{As an alternative to infinity norm, we might also use the $2$-norm, which naturally induces an inner product by Exercise~\ref{exer: condition for a norm-induced inner product}.}
% % 	\[\norm{c_1u_1+\cdots+c_nu_n}_2=\max\{\abs{c_1},\dots,\abs{c_n}\}.\]
% % 	We interpret $\norm{\cdot}_1$ as a function $(c_1,\dots,c_n)\mapsto\norm{c_1u_1+\cdots+c_nu_n}_1$ from $\F^n$ to $\R$. It suffices to show that the desired result holds on the sphere
% % 	\[S=\cbra{(c_1,\dots,c_n)\in\F^n:\norm{c_1u_1+\cdots+c_nu_n}_2=1},\]
% % 	i.e. that $\norm{\cdot}_1$ is bounded on $S$.%
% % 	\footnote{\idea We cannot manipulate two norms simultaneously. Hence we restrict to $S$ and fix an orthonormal basis. Then $V$ behaves like $\F^n$. Nice properties of $S$ lead to proof of the continuity of $\norm{\cdot}_1$.}
% % 	Because $S$ is a closed and bounded set, it suffices to show that the function $\norm{\cdot}_1$ is continuous on $\F^n$.
	
% % 	Suppose $a\in\F^n$ and $e_1,\dots,e_n$ is the standard basis of $\F^n$. As $\norm{x-a}\to0$,
% % 	\begin{align*}
% % 		\abs{\norm{x}_1-\norm{a}_1}&\leq\norm{x-a}_1=\norm{\sum_{k=1}^{n}(x_k-a_k)e_k}_1\\
% % 		&\leq\sum_{k=1}^{n}\abs{x_k-a_k}\norm{e_k}_1\\
% % 		&\leq\bra{\sum_{k=1}^{n}\norm{e_k}_1^2}^{1/2}\norm{x-a}\to0,
% % 	\end{align*}
% % 	where the last inequality follows from the Cauchy-Schwarz inequality. Because every continuous function on a closed and bounded set is bounded, the proof is completed.
% % \end{proof}
% % \begin{proof}[Proof 2]
% \begin{proof}
% 	Without loss of generality, suppose $V=\F^n$, $\norm{x}_1^2=x^\intercal Ax$, and $\norm{x}_2^2=x^\intercal Bx$, where $A,B$ are positive-definite and $B$ is diagonal.%
% 	\footnote{\idea Fixing an orthonormal basis naturally leads to a (simple) diagonal matrix.}
% 	Note that $\lambda B-A$ is Hermitian and hence diagonalizable. It suffices to show that there exists $\lambda>0$ such that $\lambda B-A$ is positive-definite. Such a $\lambda$ exists by the Gershgorin disk theorem, as desired.
% \end{proof}

% \begin{exer}
% 	Suppose $\F=\C$ and $V$ is finite-dimensional. Prove that if $T\in\L(V)$ is such that $1$ is the only eigenvalue of $T$ and $\norm{Tv}\leq\norm{v}$ for all $v\in V$, then $T$ is the identity operator.
% \end{exer}
% \begin{proof}
% 	By Schur's theorem, there exists an orthonormal basis $e_1,\dots,e_m$ of $V$ with respect to which $T$ has an upper-triangular matrix. Then all entries on the diagonal of $\M(T)$ is $1$. Any $\M(T)_{j,k}\neq0$ with $j\neq k$ would contradict $\norm{Te_k}\leq\norm{e_k}$. That proves that $\M(T)=I$, as desired.
% \end{proof}

\begin{exer}
	Suppose $v_1,\dots,v_n$ is a basis of $V$. Prove that there exists a basis $u_1,\dots,u_n$ of $V$ such that%
	\footnote{\idea Linearity in inner product and and patterns of $0,1$ inspire the use of linear functionals.}
	\[\inp{u_j}{v_k}=\left\{\begin{aligned}
		0 \quad & \text{if}\, j\neq k,\\
		1 \quad & \text{if}\, j=k.\\
	\end{aligned}\right.\]
\end{exer}
\begin{proof}
	Define $\phi_k\in V'$ for each $k$ by $\phi_k(u)=\inp{u}{v_k}$. By the Riesz representation theorem, $\phi_1,\dots,\phi_n$ is a spanning list in $V'$. Thus it is a basis of $V'$. Let $u_1,\dots,u_n$ be the basis of $V$ whose dual basis is $\phi_1,\dots,\phi_n$. Then $u_1,\dots,u_n$ satisfies the desired property.
\end{proof}


\subsection{Orthogonal Complements and Minimization Problems}
\begin{thm}[Riesz representation theorem]
	Suppose $V$ is finite-dimensional. For each $v\in V$, define $\phi_v\in V'$ by
	\[\phi_v(u)=\inp{u}{v}\]
	for each $u\in V$. Then $v\mapsto\phi_v$ is a one-to-one map from $V$ onto $V'$.%
\end{thm}
\begin{proof}[Proof Idea]
	If $\phi(u)=\inp{u}{v}$ holds for all $u\in V$, then $v\in(\ker\phi)^\perp$. However, $(\ker\phi)^\perp$ has dimension $1$ (except when $\phi=0$). Hence we can obtain the right $v$ by choosing an arbitrary nonzero $w\in(\ker\phi)^\perp$ and then multiplying by an appropriate scalar.
\end{proof}
% \begin{proof}
% 	The injectivity is trivial. We prove the surjectivity. Suppose $\phi\in V'$. The case where $\phi=0$ is trivial. Thus assume $\phi\neq0$. Hence $\ker\phi\neq V$, which implies that $(\ker\phi)^\perp\neq\{0\}$. Let $w\in(\ker\phi)^\perp$ be such that $w\neq0$. Let%
% 	\footnote{\idea Apply $w$ to $u$ to find the supposedly right scalar $c$ in $\phi(u)=\inp{u}{cw}$.}
% 	\begin{peq}\label{35peq: definition of $v$}
% 		v=\frac{\overline{\phi(w)}}{\norm{w}^2}w.
% 	\end{peq}
% 	Then $v\in(\ker\phi)^\perp$ and $v\neq0$. Now we prove that $\phi(u)=\inp{u}{v}$ for each $u\in V$.

% 	Let $u\in V$. The orthogonal decomposition leads to
% 	\[u=\frac{\inp{u}{v}}{\norm{v}^2}v+\bra{u-\frac{\inp{u}{v}}{\norm{v}^2}v},\]
% 	where the second term is orthogonal to $v$ and thus in $\ker\phi$. Applying $\phi$ to both sides, we have
% 	\begin{peq}\label{35peq: last equation}
% 		\phi(u)=\frac{\inp{u}{v}}{\norm{v}^2}\phi(v).
% 	\end{peq}
% 	By (\ref{35peq: definition of $v$}), we have
% 	\[\norm{v}=\frac{\abs{\phi(w)}}{\norm{w}} ,\qquad \phi(v)=\frac{\abs{\phi(w)}^2}{\norm{w}^2}.\]
% 	Applying them to (\ref{35peq: last equation}) leads to $\phi(u)=\inp{u}{v}$, as desired.
% \end{proof}

\begin{exer}
	Suppose $V$ is finite-dimensional and  $P\in\L(V)$ is such that $P^2=P$.
	\begin{enumerate}
		\item Suppose every vector in $\ker P$ is orthogonal to every vector in $\im P$. Prove that there exists a subspace $U$ of $V$ such that $P=P_U$.%
		\footnote{\idea Observe that $\ker P=(\im P)^\perp$. Thus we prove that $P$ and $P_{\im P}$ agree on $\ker P$ and $\im P$.}
		\item Suppose $\norm{Pv}\leq\norm{v}$ for every $v\in V$. Prove that there exists a subspace $U$ of $V$ such that $P=P_U$.%
		\footnote{\idea Observe that among all $v$'s with the same $Pv$, the one in $(\ker P)^\perp$ is the shortest. Applying this $v$ makes the best use of the inequality. Thus we prove that $P$ and $P_{(\ker P)^\perp}$ agree on $\ker P$ and $(\ker P)^\perp$.}
	\end{enumerate}
\end{exer}

\begin{props}[Algebraic properties of the pseudoinverse]
	Suppose $V$ is finite-dimensional and $T\in\L(V,W)$.
	\begin{enumerate}
		\item The pseudoinverse of an orthogonal projection is the operator itself.
		\item $\ker T^\dagger=(\im T)^\perp$ and $\im T^\dagger=(\ker T)^\perp$.
		\item $TT^\dagger T=T$ and $T^\dagger TT^\dagger=T^\dagger$.
		\item $\bra{T^\dagger}^\dagger=T$.
	\end{enumerate}
\end{props}


\section{Operators on Inner Product Spaces}
% \textbf{From now on, we suppose that $V$ and $W$ are nonzero finite-dimensional inner product spaces over $\F$.}
\subsection{Self-Adjoint and Normal Operators}
\begin{exer}
	Suppose $T\in\L(V)$ and $\lambda\in\F$. Prove that
	\begin{center}
		$\lambda$ is an eigenvalue of $T$ $\iff$ $\overline{\lambda}$ is an eigenvalue of $T^*$.
	\end{center}
\end{exer}

\begin{exer}
	Suppose $T\in\L(V)$ and $U$ is a subspace of $V$. Prove that
	\begin{center}
		$U$ is invariant under $T$ $\iff$ $U^\perp$ is invariant under $T^*$.
	\end{center}
\end{exer}

\begin{exer}
	Suppose $T\in\L(V)$ is normal. Prove that
	\begin{center}
		$\ker T^k=\ker T$ \quad and \quad $\im T^k=\im T$.
	\end{center}
\end{exer}

\begin{exer}
	Suppose $T\in\L(V,W)$. Prove that under the standard identification of $V$ with $V'$ (by Riesz representation theorem) and $W$ with $W'$, the adjoint map $T^*$ corresponds to the dual map $T'$. More precisely, prove that
	\[T'(\phi_w)=\phi_{T^*w}\]
	for all $w\in W$.
\end{exer}

\begin{rmk}
	Furthermore, under this identification of $V$ with $V'$, the orthogonal complement corresponds to the annihilator; the formulas for $\ker T^*$ and $\im T^*$ become identical to the formulas for $\ker T'$ and $\im T'$. Note that orthogonal complements and adjoints are easier to deal with than annihilators and dual maps.
\end{rmk}


\subsection{Spectral Theorem}
% \begin{thms}
% 	\begin{enumerate}
% 		\item Suppose $T\in\L(V)$. Suppose $\F=\R$. Then $T$ is self-adjoint if and only if all pairs of eigenvectors corresponding to distinct eigenvalues of $T$ are orthogonal and $V=E(\lambda_1,T)\oplus\dots\oplus E(\lambda_m,T)$, where $\lambda_1,\dots,\lambda_m$ denote the distinct eigenvalues of $T$.
% 		\item Suppose $T\in\L(V)$. Suppose $\F=\C$. Then $T$ is normal if and only if all pairs of eigenvectors corresponding to distinct eigenvalues of $T$ are orthogonal and $V=E(\lambda_1,T)\oplus\dots\oplus E(\lambda_m,T)$, where $\lambda_1,\dots,\lambda_m$ denote the distinct eigenvalues of $T$.
% 	\end{enumerate}
% \end{thms}

\begin{exer}
	\begin{enumerate}
		\item Suppose $\F=\C$ and $\E\subseteq\L(V)$. Prove that there is an orthonormal basis of $V$ with respect to which every element of $\E$ has a diagonal matrix if and only if $S$ and $T$ are commuting normal operators for all $S,T\in\E$.
		\item Suppose $\F=\R$ and $\E\subseteq\L(V)$. Prove that there is an orthonormal basis of $V$ with respect to which every element of $\E$ has a diagonal matrix if and only if $S$ and $T$ are commuting self-adjoint operators for all $S,T\in\E$.%
		\footnote{This exercise extends the spectral theorems to the context of a collection of commuting operators.}
	\end{enumerate}
\end{exer}


\subsection{Positive Operators}
\begin{exer}
	For $T\in\L(V)$ and $u,v\in V$, define $\inp{u}{v}_T$ by $\inp{u}{v}_T=\inp{Tu}{v}$.
	\begin{enumerate}
		\item Suppose $T\in\L(V)$. Prove that $\inp{\cdot}{\cdot}_T$ is an inner product on $V$ if and only if $T$ is an invertible positive operator (with respect to the original inner product).
		\item Prove that every inner product on $V$ is of the form $\inp{\cdot}{\cdot}_T$ for some invertible positive operator $T\in\L(V)$.
	\end{enumerate}
\end{exer}
\begin{proof}[Proof Idea]
	(b) Let $\inp{\cdot}{\cdot}_1$ be an inner product on $V$. We need to find $T$ such that $\inp{u}{v}_1=\inp{Tu}{v}$. Fixing $v$ is not sufficient to find the unique $Tu$. Thus we fix $u$ and consider $\inp{v}{u}_1=\inp{v}{Tu}$. The map $v\mapsto\inp{v}{u}_1$ is in $V'$. Hence we use the Riesz representation theorem to define $Tu$.
\end{proof}


\subsection{Singular Value Decomposition}
\begin{exer}
	Suppose $T\in\L(V,W)$ and $s>0$. Prove that $s$ is a singular value of $T$ if and only if there exist nonzero vectors $v\in V$ and $w\in W$ such that%
	\footnote{The vectors $v,w$ satisfying both equations above are called a \emph{Schmidt pair}.}
	\[Tv=sw\quad\text{and}\quad T^*w=sv.\]
\end{exer}

\begin{exer}
	Suppose $T\in\L(V,W)$. Suppose $s_1\geq s_2\geq\dots\geq s_m>0$ and $e_1,\dots,e_m$ is an orthonormal list in $V$ and $f_1,\dots,f_m$ is an orthonormal list in $W$ such that
	\[Tv=s_1\inp{v}{e_1}f_1+\dots+s_m\inp{v}{e_m}f_m\]
	for every $v\in V$.
	\begin{enumerate}
		\item Prove that $s_1,\dots,s_m$ are the positive singular values of $T$.
		\item Prove that if $k\in\{1,\dots,m\}$, then $e_k$ is an eigenvector of $T^*T$ with corresponding eigenvalue $s_k^2$.
	\end{enumerate}
\end{exer}

\begin{exer}
	Suppose $T\in\L(V,W)$. Let $n=\dim V$ and let $s_1\geq\dots\geq s_n$ denote the singular values of $T$. Prove that if $1\geq k\geq n$, then
	\[\min\{\norm{\rest{T}{U}}:U\text{ is a subspace of $V$ with $\dim U=k$}\}=s_{n-k+1}.\]
\end{exer}
\begin{rmk}
	Compare the result above to that
	\[\min\{\norm{T-S}:S\in\L(V,W)\text{ and $\dim\im S\leq k$}\}=s_{k+1}.\]
\end{rmk}

\begin{exer}
	Suppose $T\in\L(V)$, $S\in\L(V)$ is a unitary operator, and $R\in\L(V)$ is a positive operator such that $T=SR$. Prove that $R=\sqrt{T^*T}$.
\end{exer}


\section{Operators on Complex Vector Spaces}
\subsection{Generalized Eigenvectors and Nilpotent Operators}
\begin{exer}
	Suppose $T\in\L(V)$. $m\in\Z^+$ is such that the minimal polynomial of $T$ is a polynomial multiple of $(z-\lambda)^m$. Prove that
	\[\dim\ker(T-\lambda I)^m\geq m.\]
\end{exer}
\begin{proof}
	Let the minimal polynomial of $T$ be $(z-\lambda)^mq(z)$. Observe that $m$ is the least integer (If $k<m$ satisfies the property then $(T-\lambda I)^kq(T)=0$.) such that
	\[\rest{(T-\lambda I)^m}{\im q(T)}=0.\]
	Thus $T-\lambda I$ is nilpotent on $\im q(T)$, which implies that $m\leq\dim\im q(T)$.

	On the other hand,
	\[\rest{q(T)}{\im(T-\lambda I)^m}=0.\]
	Thus $\ker q(T)\supseteq\im(T-\lambda I)^m$. Hence
	\[\dim\ker(T-\lambda I)^m=n-\dim\im(T-\lambda I)^m\geq n-\dim\ker q(T)=\dim\im q(T)\geq m.\]
\end{proof}

\begin{prop}
	Suppose that $T\in\L(V)$. Then there is a basis consisting of generalized eigenvectors of $T$ if and only if the minimal polynomial of $T$ equals $(z-\lambda_1)\dots(z-\lambda_m)$ for some $\lambda_1,\dots,\lambda_m\in\F$.
\end{prop}


\subsection{Generalized Eigenspace Decomposition}
\begin{exer}
	Suppose $T\in\L(V)$ and $\lambda$ is an eigenvalue of $T$. Prove that the exponent of $z-\lambda$ in the factorization of the minimal polynomial of $T$ is the least positive integer $m$ such that $\rest{(T-\lambda I)^m}{G(\lambda,T)}=0$.
\end{exer}

\begin{exer}
	Suppose $T\in\L(V)$ and $\lambda$ is an eigenvalue of $T$ with multiplicity $d$. Prove that $G(\lambda, T)=\ker(T-\lambda I)^d$.
\end{exer}

\begin{exer}
	Suppose $\F=\C$ and $T\in\L(V)$. Prove that there exist $D,N\in\L(V)$ such that $T=D+N$, $D$ is diagonalizable, $N$ is nilpotent, and $DN=ND$.%
	\footnote{The generalized eigenspace decomposition allows us to restrict our attention to a single generalized eigenspace.}
\end{exer}


\subsection{Trace: A Connection Between Matrices and Operators}
\begin{exer}
	Prove that the trace is the only linear functional $\tau:\L(V)\to\F$ such that
	\[\tau(ST)=\tau(TS)\]
	for all $S,T\in\L(V)$ and $\tau(I)=\dim V$.
\end{exer}
\begin{proof}[Proof Idea]
	Define $P_{j,k}\in\L(V)$ by $P_{j,k}(a_1v_1+\dots+a_nv_n)=a_kv_j$, where $v_1,\dots,v_n$ is a basis of $V$. Prove that $\tau(P_{j,k})=1$ if $j=k$ and $0$ otherwise.
\end{proof}
\begin{proof}
	Taking $S=P_{j,j},T=P_{j,k}$ leads to the desired result for $j\neq k$. For $P_{j,j}$, consider its matrix version and  use $\tau(D)=\tau(P^{-1}DP)$ where $P$ is a permutation matrix.
\end{proof}

\begin{exer}
	Suppose $V$ and $W$ are finite-dimensional inner product spaces.
	\begin{enumerate}
		\item Prove that
		\[\tr(T^*T)=\sum_{k=1}^{n}\sum_{j=1}^{m}\abs{\inp{Te_k}{f_j}}^2,\]
		where $e_1,\dots,e_n$ is an orthonormal basis of $V$ and $f_1,\dots,f_m$ is an orthonormal basis of $W$.
		\item Prove that $\inp{S}{T}=\tr(T^*S)$ defines an inner product on $\L(V,W)$, and it is the same as the standard inner product on $\F^{mn}$.
	\end{enumerate}
\end{exer}
\begin{rmks}
	The right-hand side of (a) is the sum of the squares of the absolute values of the matrix entries, and does not depend on the chosen orthonormal bases. The norm on $\L(V,W)$ induced by (b) is called the \emph{Frobenius norm} or the \emph{Hilbert-Schmidt norm}.
\end{rmks}


\section{Multilinear Algebra and Determinants}
\subsection{Determinants}
\begin{exer}
	Suppose $n$ is a positive integer and $\delta:\C^{n,n}\to\C$ is a function such that
	\[\delta(AB)=\delta(A)\cdot\delta(B)\]
	for all $A,B\in\C^{n,n}$ and $\delta(A)$ equals the product of the diagonal entries of $A$ for each diagnonal matrix $A\in\C^{n,n}$. Prove that
	\[\delta(A)=\det A\]
	for all $A\in\C^{n,n}$.%
	\footnote{\idea Use the Gaussian elimination and diagonalization.}
\end{exer}


\subsection{Tensor Products}
\begin{exer}
	Suppose $m$ and $n$ are positive integers. For $v\in\F^m$ and $w\in\F^n$, identify $v\otimes w$ with an $m$-by-$n$ matrix. With that identification, show that the set
	\[\{v\otimes w:v\in\F^m\text{ and }w\in\F^n\}\]
	is the set of $m$-by-$n$ matrices that have rank at most one.
\end{exer}

\begin{exer}
	Suppose $\dim V>1$ and $\dim W>1$. Prove that
	\[\{v\otimes w:(v,w)\in(V,W)\}\]
	is not a subspace of $V\otimes W$.
\end{exer}


\section{Complexifcation}
\begin{props}[Properties of complexification]
	\begin{enumerate}
		\item $\lambda\in\R$ is an eigenvalue of $T$ if and only if $\lambda$ is an eigenvalue of $T_{\C}$.
		\item $\lambda\in\C$ is an eigenvalue of $T_{\C}$ if and only if $\overline{\lambda}$ is an eigenvalue of $T_{\C}$.
		\item The minimal polynomial of $T_{\C}$ equals the minimal polynomial of $T$.
		\item Suppose $V$ is a real inner product space. For $u,v,w,x\in V$, define
		\[\inp{u+iv}{w+ix}_{\C}=\inp{u}{w}+\inp{v}{x}+(\inp{v}{w}-\inp{u}{x})i.\]
		Then $\inp{\cdot}{\cdot}_{\C}$ makes $V_{\C}$ into a complex inner product space. If $u,v\in V$, then
		\[\inp{u}{v}_{\C}=\inp{u}{v}\quad\text{and}\quad\norm{u+iv}_{\C}^2=\norm{u}^2+\norm{v}^2.\]
	\end{enumerate}
\end{props}

\begin{prop}
	Every operator on a finite-dimensional nonzero vector space has an invariant subspace of dimension $1$ or $2$.
\end{prop}
\begin{proof}
	The case where $\F=\C$ is trivial. Now assume $V$ is a real vector space and $T\in\L(V)$. Then $T_{\C}$%
	\footnote{\idea Field extension.}
	has an eigenvalue $a+bi$ with $a,b\in\R$. Thus there exist $u,v\in V$, not both $0$, such that
	\[Tu+iTv=(au-bv)+(av+bu)i.\]
	Hence $\spn(u,v)$ is invariant under $T$, as desired.
\end{proof}

\begin{exer}
	Suppose $\F=\R$, $T\in\L(V)$, and $\lambda\in\C$.
	\begin{enumerate}
		\item Prove that $u+iv\in G(\lambda,T_\C)$ if and only if $u-iv\in G(\overline{\lambda}, T_\C)$.
		\item Prove that the multiplicity of $\lambda$ as an eigenvalue of $T_\C$ equals that of $\overline{\lambda}$.
		\item Use (b) to show that if $\dim V$ is an odd number, then $T_\C$ has a real eigenvalue.
		\item Use (c) to show that if $\dim V$ is an odd number, then $T$ has an eigenvalue.
	\end{enumerate}
\end{exer}


\section{Primary Decomposition}
\begin{exer}[Primary Decomposition]
	\begin{enumerate}
		\item Suppose $f,g$ are coprime polynomials and $T\in\L(V)$. Prove that%
		\footnote{\idea Use Bezout's lemma.}
		\[\ker f(T)\oplus\ker g(T)=\ker(fg)(T).\]
		\item Suppose $T\in\L(V)$ and the minimal polynomial of $T$
		\[p=p_1^{\alpha_1}p_2^{\alpha_2}\dots p_m^{\alpha_m},\]
		where $p_k$'s are pairwise coprime. Prove that
		\[V=\bigoplus_{k=1}^m\ker p_k^{\alpha_k}(T).\]
	\end{enumerate}
\end{exer}

\begin{exer}
	Suppose $\F=\C$. Prove the generalized eigenspace decomposition, and prove that
	\[G(\lambda,T)=\ker(T-\lambda I)^m,\]
	where $T\in\L(V)$, $\lambda\in\F$, and $m$ is the exponent of $z-\lambda$ in the factorization of the minimal polynomial of $T$.
\end{exer}

\begin{exer}
	Suppose $V$ is a finite-dimensional inner product space. Suppose $P\in\L(V)$ is such that $P^2=P$ and every vector in $\ker P$ is orthogonal to every vector in $\im P$. Prove that there exists a subspace $U$ of $V$ such that $P=P_U$.
\end{exer}


\end{document}