\documentclass{tufte-handout}
\usepackage{amsmath,amsfonts,amsthm,amssymb}
\usepackage{bm}
\usepackage{pgf,tikz}
\usepackage{float}
\usepackage{multicol}
\usepackage{booktabs}
\usepackage{enumerate,enumitem}
\usepackage{setspace}
\usepackage{fancyhdr} % custom headers and footers
\usepackage{ragged2e}
\usepackage{sectsty} % Allow customizing section commands


% ---------- Preamble ----------
\pagestyle{fancyplain} % Make all pages in the document conform to the custom headers and footers
\fancyhead{} % No page header
\fancyfoot[L]{}
\fancyfoot[C]{}
\fancyfoot[R]{\thepage}
\renewcommand{\headrulewidth}{0pt} % Remove header underlines
\renewcommand{\footrulewidth}{0pt} % Remove footer underlines
\setlength{\headheight}{13.6pt} % Customize the height of the header
\allowdisplaybreaks
\geometry{
	left=13mm, % left margin
	textwidth=130mm, % main text block
	marginparsep=8mm, % gutter between main text block and margin notes
	marginparwidth=55mm % width of margin notes
}
\setlist[enumerate]{
	label=(\alph*),
	leftmargin=2.5em,
	topsep=0.5em,
	itemsep=0em,
}
\def \v {\vspace{0.2cm}}
\allsectionsfont{\normalfont\bfseries} % Make all sections centered, the default font and small caps


% ---------- Theorem Environments ----------
\theoremstyle{plain} % default
\newtheorem{thm}{Theorem}
\newtheorem{cor}[thm]{Corollary}
\newtheorem{prop}[thm]{Proposition}
\newtheorem{props}[thm]{Propositions}
\newtheorem{lem}[thm]{Lemma}
\newtheorem{conj}[thm]{Conjecture}
\newtheorem{quest}[thm]{Question}

\theoremstyle{definition}
\newtheorem{defn}[thm]{Definition}
\newtheorem{defns}[thm]{Definitions}
\newtheorem{exmp}[thm]{Example}
\newtheorem{exmps}[thm]{Examples}
\newtheorem{notn}[thm]{Notation}
\newtheorem{notns}[thm]{Notations}
\newtheorem{exer}[thm]{Exercise}

\theoremstyle{remark}
\newtheorem{rem}[thm]{Remark}
\newtheorem{rems}[thm]{Remarks}


% ---------- Symbol Macros ----------
\newcommand{\bra}[1]{\mathopen{}\left(#1\right)}
\newcommand{\sbra}[1]{\mathopen{}\left[#1\right]}
\newcommand{\cbra}[1]{\mathopen{}\left\{#1\right\}}
\newcommand{\norm}[1]{\left\lVert #1 \right\rVert}
\newcommand{\inp}[2]{\left\langle #1, #2 \right\rangle}
\newcommand{\abs}[1]{\left| #1 \right|}
\newcommand{\rest}[2]{\mathopen{}\left.#1\right|_{#2}}

\renewcommand{\phi}{\varphi}

\newcommand{\R}{\mathbb{R}}
\newcommand{\N}{\mathbb{N}}
\newcommand{\Z}{\mathbb{Z}}
\newcommand{\C}{\mathbb{C}}
\newcommand{\Q}{\mathbb{Q}}
\newcommand{\F}{\mathbb{F}}
\renewcommand{\L}{\mathcal{L}}
\newcommand{\M}{\mathcal{M}}
\renewcommand{\P}{\mathcal{P}}
\newcommand{\B}{\mathcal{B}}
\newcommand{\E}{\mathcal{E}}
\newcommand{\zero}{\mathbf{0}}

\renewcommand{\intercal}{t}
\newcommand{\e}{\mathrm{e}}

\DeclareMathOperator{\dist}{dist}
\DeclareMathOperator{\spn}{span}
\DeclareMathOperator{\im}{im}
% \DeclareMathOperator{\ker}{ker} % already existed
\DeclareMathOperator{\tr}{tr}
\DeclareMathOperator{\rank}{rank}


% ---------- Title Section ----------
\title{	
	\normalfont\normalsize 
	{\itshape Linear Algebra Done Right} \\ [0pt]
	\huge Notes -- Linear Algebra
}
\author{Zhijie Chen}
\date{\vspace{-5pt}\normalsize\today}


\begin{document}
\justifying
\maketitle
\tableofcontents
\newpage

\thispagestyle{empty}
\begin{fullwidth}
	\Large
	\setstretch{1.5}
	\vspace*{\fill}
	\begin{center}
		Don't just read it; fight it!\\
		Ask your own questions.\\
		Look for your own examples.\\
		Discover your own proofs.\\
		Is the hypothesis necessary?\\
		Is the converse true?\\
		What happens in the classical special case?\\
		What about the degenerate cases?\\
		Where does the proof use the hypothesis?
	\end{center}
	\begin{flushright}
		\textemdash\,Paul Holmos\phantom{placeholderrrrrr}
	\end{flushright}
	\vspace*{\fill}
\end{fullwidth}
\newpage


% \section{Prerequisites}
% We declare several notations below for convenience and clarity.
% \begin{notns}[Basic notations]
% 	\phantom{linebreak}

%     \begin{itemize}
%         \item $\F$ denotes a number field.
%         \item $U$, $V$, $W$ denotes vector spaces (usually over scalar field $\F$).
%         \item $V^S$ denotes the set of functions from a nonempty set $S$ to a vector space $V$.
%     \end{itemize}
% \end{notns}


\section{Vector Spaces}
\begin{defn}
	The complexification of $V$, denoted by $V_\C$, equals $V\times V$ with normal addition and real scalar multiplication for product space. But we write an element $(u,v)$ of $V_\C$ as $u+iv$. Complex scalar multiplication is defined by
	\[(a+bi)(u+iv)=(au-bv)+i(av+bu)\]
	for all $a,b\in\R$ and all $u,v\in V$.%
	\footnote{Think of $V$ as a subset of $V_\C$ by identifying $u\in V$ with $u+i0$. The construction of $V_\C$ from $V$ can then be thought of as generalizing the construction of $\C^n$ from $\R^n$.}
\end{defn}

\begin{lem}[Linear dependence lemma]
	Suppose $v_1,\dots,v_m$ is a linearly dependent set in $V$. Then there exists $k\in\{1,2,\dots,m\}$ such that
	\[v_k\in\spn(v_1,\dots,v_m).\]
	Furthermore, removing the $k^\text{th}$ term from the list does not change the span.%
	\footnote{This lemma lays the foundation for a series of basic results for vector spaces.}
\end{lem}

\begin{thm}
	Any two bases of a finite-dimensional vector space have the same length.%
	\footnote{This proposition ensures that the \emph{dimension} of a vector space is well-defined.}
\end{thm}
\begin{proof}
	Suppose $V$ is finite-dimensional. Let $\B_1$ and $\B_2$ be two bases of $V$. Considering $\B_1$ as an independent set and $\B_2$ as a spanning set leads to $\#\B_1\leq\#\B_2$. Interchanging the roles of $\B_1$ and $\B_2$ and we have $\#\B_2\leq\#\B_1$. Thus $\#\B_1=\#\B_2$.
\end{proof}


\section{Linear Maps}
\subsection{Kernal and Image of Linear Maps}
\begin{exer}
	Suppose $U$ and $V$ are finite-dimensional and $S\in\L\bra{V,W}$ and $T\in\L\bra{U,V}$. Prove that
	\[\dim\ker ST \leq \dim\ker S + \dim\ker T.\]
\end{exer}
\begin{proof}
	Restrict to $Z=\ker ST$. By the fundamental theorem of linear maps,
	\begin{align*}
		\dim Z & = \dim T(Z) + \dim\ker \rest{T}{Z} \\
		& \leq \dim T(Z) + \dim\ker T \\
		& = \dim ST(Z) + \dim\ker \rest{S}{T(Z)} + \dim\ker T \\
		& \leq \dim\ker S + \dim\ker T.\qedhere
	\end{align*}
\end{proof}

\begin{cor}[Sylvester's rank inequality]
	Suppose $A\in\F^{m,n}$ and $B\in\F^{n,p}$ are two matrices. Then%
	\footnote{There is a slicker proof for this inequality using block matrices. But the proof here using linear maps is more informative.}
	\[\rank A+\rank B-n\leq\rank\bra{AB}.\]
\end{cor}


\subsection{Products and Quotients of Vector Spaces}
% \begin{lem}\label{lem: direct sum and product space}
% 	Suppose $V_1,\dots,V_m$ are subspaces of $V$.%
%     \footnote{Note that $V$ does not have to be finite-dimensional. Recall that $V_1+\cdots+V_m$ is a direct sum if and only if the only way to write $0$ as a sum of $v_1+\cdots+v_m$, where each $v_k\in V_k$, is by taking each $v_k$ equal to $0$.}
% 	Define a linear map $\Gamma: V_1\times\cdots\times V_m \to V_1+\cdots+V_m$ by
% 	\[\Gamma\bra{v_1,\dots,v_m} = v_1+\cdots+v_m.\]
% 	Then $V_1+\cdots+V_m$ is a direct sum if and only if $\Gamma$ is injective.
% \end{lem}

% \begin{thm}
% 	Suppose $V$ is finite-dimensional and $V_1,\dots,V_m$ are subspaces of $V$. Then $V_1+\cdots+V_m$ is a direct sum if and only if%
%     \footnote{Recall Lemma~\ref{lem: direct sum and product space}.}
% 	\[\dim\bra{V_1+\cdots+V_m} = \dim V_1+\cdots+\dim V_m.\]
% \end{thm}
% \begin{proof}
% 	Recall Lemma~\ref{lem: direct sum and product space}.Because $\Gamma$ is surjective, by the fundamental theorem of linear maps, $V_1+\cdots+V_m$ is a direct sum if and only if $\dim\bra{V_1+\cdots+V_m} = \dim\bra{V_1\times\cdots\times V_m} = \dim V_1+\cdots+\dim V_m$.
% \end{proof}

\begin{notn}
	Suppose $T\in\L\bra{V,W}$. Define $\widetilde{T}: V+\ker V \to V$ by
	\[\widetilde{T}\bra{v+\ker T}=Tv.\]
\end{notn}

\begin{exer}
	Suppose $V_1,\dots,V_m$ are vector spaces. Prove that $\L\bra{V_1\times\cdots\times V_m,W}$ and $\L\bra{V_1,W}\times\cdots\times\L\bra{V_m,W}$ are isomorphic vector spaces. 
\end{exer}
\begin{proof}
	We construct an isomorphism $T$ between the two vector spaces.
	
	For every $\Gamma\in\L\bra{V_1\times\cdots\times V_m,W}$, define $\phi_k:V_k\to W$ for each $k$ by
	\[\phi_k\bra{v_k}=\Gamma\bra{0,\dots,v_k,\dots,0}\]
	with $v_k$ in the $k^\text{th}$ slot and $0$ in all other slots. It can be verified that $\phi_k\in\L\bra{V_k,W}$.

	Define $T$ by $T\bra{\Gamma}=\bra{\phi_1,\dots,\phi_m}$. It can be verified that $T$ is a linear map. We prove $T$ is an isomorphism by constructing its inverse linear map $S$.

	For every $\bra{\phi_1,\dots,\phi_m}\in\L\bra{V_1,W}\times\cdots\times\L\bra{V_m,W}$, let
	\[S\bra{\phi_1,\dots,\phi_m}\bra{v_1,\dots,v_m}=\phi_1\bra{v_1}+\cdots+\phi_m\bra{v_m}.\]
	
	It can be shown that $S$ is a linear map, and that $S\circ T=I$ and $T\circ S=I$. That proves $T$ is indeed an isomorphism between the two vector spaces.
\end{proof}

\begin{prop}\label{prop: test for translate}
	A nonempty subset $A$ of $V$ is a translate of some subspace of $V$ if and only if $\lambda v+\bra{1-\lambda}w\in A$ for all $v,w\in A$ and all $\lambda\in\F$.
\end{prop}

\begin{exer}
	Suppose $A_1=v+U_1$ and $A_2=w+U_2$ for some $v,w\in V$ and some subspaces $U_1$, $U_2$ of $V$. Prove that $A_1\cap A_2$ is either the empty set or a translate of some subspace of $V$.%
    \footnote{Recall Proposition~\ref{prop: test for translate}.}
\end{exer}

\begin{prop}\label{prop: direct sum from quotient space}
	Suppose $U$ is a subspace of $V$ and $v_1+U,\dots,v_m+U$ is a basis of $V/U$ and $u_1,\dots,u_n$ is a basis of $U$. Then $v_1,\dots,v_m,u_1,\dots,u_n$ is a basis of $V$. In other words, $V=\spn(v_1,\dots,v_m)\oplus U$.%
    \footnote{$V=\spn(v_1,\dots,v_m)\oplus U$ still holds without the hypothesis that $U$ is finite-dimensional.}
\end{prop}

\begin{exer}
	Suppose $U$ is a subspace of $V$ such that $V/U$ is finite-dimensional.
	\begin{enumerate}
		\item Prove that if $W$ is a finite-dimensional subspace of $V$ and $V=U+W$, then $\dim W\geq\dim V/U$.
		\item Prove that there exists a finite-dimensional subspace $W$ of $V$ such that $V=U\oplus W$ and $\dim W=\dim V/U$.
	\end{enumerate}
\end{exer}
\begin{proof}
	Let $\overline{w}_1+U,\dots,\overline{w}_m+U$ be a basis of $V/U$. Then by Proposition~\ref{prop: direct sum from quotient space}, we have $V=\spn(\overline{w}_1,\dots,\overline{w}_m)\oplus U$. Let $W_0=\spn(\overline{w}_1,\dots,\overline{w}_m)$, then $V=U\oplus W_0$, as desired.

	Now we prove that for each subspace $W$ of $V$ such that $V=U+W$, we have $\dim W\geq m=\dim V/U$.
	
	For each $\overline{w}_i\in V$ above, by definition we have $\overline{w}_i=u_i+w_i$ for some $u_i\in U$ and $w_i\in W$. It can be shown from the linear independence of $\overline{w}_1+U,\dots,\overline{w}_m+U$ that $\overline{w}_1-u_1,\dots,\overline{w}_m-u_m$ are independent vectors in $W$. Hence $\dim W\geq m$.
\end{proof}


\subsection{Duality}
\begin{thm}
	Suppose $V$ and $W$ are finite-dimensional and $T\in\L\bra{V,W}$. Then
	\begin{center}
	$T$ is surjective $\iff T'$ is injective \quad and \quad $T$ is injective $\iff T'$ is surjective.%
    \footnote{This result can be useful because sometimes it is easier to verify that $T'$ is injective (surjective) than to show directly that $T$ is surjective (injective).}
	\end{center}
\end{thm}

\begin{prop}\label{prop: relation between subspace and annihilator}
	Suppose $V$ is finite-dimensional and $U$ is a subspace of $V$. Then
	\[U=\cbra{v\in V:\phi(v)=0\,\text{ for every }\,\phi\in U^0}.\]
\end{prop}

\begin{exer}
	Suppose $V$ is finite-dimensional and $U$ and $W$ are subspaces of $V$.
	\begin{enumerate}
		\item Prove that $W^0\subseteq U^0$ if and only if $U\subseteq W$.
		\item Prove that $W^0=U^0$ if and only if $U=W$.%
    	\footnote{Recall Proposition~\ref{prop: relation between subspace and annihilator}.}
	\end{enumerate}
\end{exer}

\begin{exer}
	Suppose $V$ is finite-dimensional and $U$ and $W$ are subspaces of $V$.\
	\begin{enumerate}
		\item Prove that $\bra{U+W}^0=U^0\cap W^0$.
		\item Prove that $\bra{U\cap W}^0=U^0+W^0$.
	\end{enumerate}
\end{exer}

\begin{prop}\label{prop: matrix of dual basis}
	Suppose $V$ is finite-dimensional and $v_1,\dots,v_n$ is a basis of $V$. Then $\phi_1,\dots,\phi_n\in V'$ is the dual basis of $v_1,\dots,v_n$ if and only if
	\[\begin{bmatrix}
		\M(\phi_1,(v_1,\dots,v_n))\\
		\vdots\\
		\M(\phi_n,(v_1,\dots,v_n))\\
	\end{bmatrix}=I.\]
	% where $\M\bra{\phi_i}$ is the $1\times n$ matrix of $\phi_i$ with respect to basis $v_1,\dots v_n$ of $V$ for each $i\in\cbra{1,\dots,n}$.
\end{prop}

\begin{exer}
	Suppose $V$ is finite-dimensional and $\phi_1,\dots,\phi_n$ is a basis of $V'$. Prove that there exists a basis of $V$ whose dual basis is $\phi_1,\dots,\phi_n$.
\end{exer}
\begin{proof}
	We start from an arbitrary basis $u_1,\dots,u_n$ of $V$. Let $\psi_1,\dots,\psi_n$ be its dual basis. In this proof, we take standard basis $e_1,\dots,e_n$ as the basis of $\F^n$.
	
	Define $S,T\in\L\bra{V,\F^n}$ by
	\[T(v)=\bra{\phi_1(v),\dots,\phi_n(v)},\quad S(v)=\bra{\psi_1(v),\dots,\psi_n(v)}.\]
	Then by Proposition~\ref{prop: matrix of dual basis}, $\M\bra{S,\bra{u_1,\dots,u_n}}=I$.
	
	Let $A$ be the change of basis matrix from $\psi$'s to $\phi$'s, i.e.,
	% \[A=\M\bra{I,\bra{\phi_1,\dots,\phi_n},\bra{\psi_1,\dots,\psi_n}}.\]
	\[\begin{bmatrix}\phi_1&\cdots&\phi_n\end{bmatrix}=\begin{bmatrix}\psi_1&\cdots&\psi_n\end{bmatrix}A.\]
	Then by the definition of change of basis matrix, we have
	\begin{align*}
		\M\bra{T,\bra{u_1,\dots,u_n}}&=\begin{bmatrix}
			\M\bra{\phi_1,\bra{u_1,\dots,u_n}}\\
			\vdots\\
			\M\bra{\phi_n,\bra{u_1,\dots,u_n}}\\
		\end{bmatrix}
		=A^\intercal\begin{bmatrix}
			\M\bra{\psi_1,\bra{u_1,\dots,u_n}}\\
			\vdots\\
			\M\bra{\psi_n,\bra{u_1,\dots,u_n}}\\
		\end{bmatrix}\\\\
		&=A^\intercal\cdot\M\bra{S,\bra{u_1,\dots,u_n}}=A^\intercal.
	\end{align*}

	Consider basis $v_1,\dots,v_n$ of $V$ such that the change of basis matrix from $u$'s to $v$'s is $\bra{A^\intercal}^{-1}$.%
    \footnote{The change of basis for $V'\to V'$ corresponds to the transpose of $V\leftarrow V$, where transpose and inverse both come from duality. That gives the idea of considering $\bra{A^\intercal}^{-1}$.}
	Thus
	\[\M\bra{T,\bra{v_1,\dots,v_n}}=\M\bra{T,\bra{u_1,\dots,u_n}}\cdot\M\bra{I,\bra{v_1,\dots,v_n},\bra{u_1,\dots,u_n}}=I.\]
	Then by Proposition~\ref{prop: matrix of dual basis}, the dual basis of $v_1,\dots,v_n$ is precisely $\phi_1,\dots,\phi_n$, as desired.
\end{proof}

\begin{exer}[A natural isomorphism from primal space onto double dual space]
	Define $\Lambda:V\to V''$ by
	\[(\Lambda v)(\phi)=\phi(v)\]
	for each $v\in V$ and $\phi\in V'$.
	\begin{enumerate}
		\item Prove that if $T\in\L(V)$, then $T''\circ\Lambda=\Lambda\circ T$.
		\item Prove that if $V$ is finite-dimensional, then $\Lambda$ is an isomorphism from $V$ onto $V''$.%
    	\footnote{Suppose $V$ is finite-dimensional. Then $V$ and $V'$ are isomorphic, but finding an isomorphism from $V$ onto $V'$ generally requires choosing a basis of $V$. In contrast, the isomorphism $\Lambda$ from $V$ onto $V''$ does not require a choice of basis and thus is more natural.}%
		\footnote{Another natural isomorphism is $\pi'\in\L\bra{(V/U)',V'}$ where $\pi$ is the normal quotient map.}
	\end{enumerate}
\end{exer}


\section{Polynomials}
\begin{thm}
	Suppose $p\in\P(\F)$ is a nonconstant polynomial of degree $m$. Then $\lambda\in\F$ is a zero of $p$ if and only if there exists a polynomial $q\in\P(\F)$ of degree $m-1$ such that $p(z)=(z-\lambda)q(z)$ for every $z\in\F$.
\end{thm}

\begin{thm}
	Suppose $p\in\P(\F)$ is a nonconstant polynomial of degree $m$. Then $p$ has at most $m$ zeros in $\F$.%
	\footnote{This theorem implies that when a polynomial $p$ has too many zeros, $p=0$.}%
    \footnote{This theorem implies that the coefficients of a polynomial are uniquely determined. In particular, the \emph{degree} of a polynomial is well-defined.}
\end{thm}

\begin{thm}[Division algorithm for polynomials]
	Suppose that $p,s\in\P(\F)$, with $s\neq0$. Then there exist unique polynomials $q,r\in\P(\F)$ such that $p=sq+r$.%
	% \footnote{The division algorithm for polynomials can be proved without using any linear algebra. This proof makes a nice use of a basis of $\P_n(\F)$.}
\end{thm}
\begin{proof}
	Let $n=\deg p$ and $m=\deg s$. The case where $n<m$ is trivial. Thus we now assume that $n\geq m$.

	The list
	\[1,z,\dots,z^{m-1},s,zs,\dots,z^{n-m}s\]
	is linearly independent in $\P_n(\F)$. And it also has length $n+1$. Hence the list is a basis of $\P_n(\F)$.

	Because $p\in\P_n(\F)$, there exist unique constants $a_0,\dots,a_{m-1},b_0,\dots,b_{n-m}\in\F$ such that
	\begin{align*}
		p&=a_0+a_1z+\cdots+a_{m-1}z^{m-1}+b_0s+b_1zs+\cdots+b_{n-m}z^{n-m}s\\
		&=\bra{a_0+a_1z+\cdots+a_{m-1}z^{m-1}}+s\bra{b_0+b_1z+\cdots+b_{n-m}z^{n-m}}.\qedhere
	\end{align*}
\end{proof}

\begin{thm}[Fundamental theorem of algebra, first version]
	Every nonconstant polynomial with complex coefficients has a zero in $\C$.
\end{thm}
\begin{proof}
	Suppose $p\in\P(\C)$ is a nonconstant polynomial with highest-order nonzero term $c_mz^m$. Then $\abs{p(z)}\to\infty$ as $\abs{z}\to\infty$. Thus the continuous function $z\mapsto\abs{p(z)}$ has a global minimum at some $\zeta\in\C$. Assume that $p(\zeta)\neq0$.

	Consider polynomial $q(z)=p(z+\zeta)/p(\zeta)$. The function $z\mapsto\abs{q(z)}$ has a global minimum at $z=0$. Write
	\[q(z)=1+a_kz_k+\cdots+a_mz^m\]
	where $k$ is the smallest positive integer such that the coefficient of $z_k$ is nonzero.
	
	Let $\beta$ be a $k^\text{th}$ root of $-1/a_k$. There is a constant $c>1$ such that if $t\in(0,1)$, then
	\[\abs{q(t\beta)}\leq\abs{1+a_kt^k\beta^k}+ct^{k+1}=1-t^k(1-tc).\]
	Thus taking $t$ to be $1/(2c)$ leads to $\abs{q(t\beta)}<1$. The contradiction implies that $p(\zeta)=0$, as desired.
\end{proof}

\begin{thm}[Fundamental theorem of algebra, second version]
	If $p\in\P(\C)$ is a nonconstant polynomial, then $p$ has a unique factorization of the form
	\[p(z)=c(z-\lambda_1)\cdots(z-\lambda_m),\]
	where $c,\lambda_1,\dots,\lambda_m\in\C$.
\end{thm}

\begin{thm}[Factorization of a polynomial over $\R$]
	If $p\in\P(\R)$ is a nonconstant polynomial, then $p$ has a unique factorization of the form
	\[p(x)=c(x-\lambda_1)\cdots(x-\lambda_m)(x^2+b_1x+c_1)\cdots(x^2+b_Mx+c_M),\]
	where $m,M\in\N$ and $c,\lambda_1,\dots,\lambda_m,b_1,\dots,b_M,c_1,\dots,c_M\in\R$, with $b_k^2<4c_k$ for each $k$.
\end{thm}

\begin{exer}
	Suppose $p,q\in\P(\C)$ are nonconstant polynomials with no zeros in common. Let $m=\deg p$ and $n=\deg q$. Prove that there exist $r\in\P_{n-1}(\C)$ and $s\in\P_{m-1}(\C)$ such that $rp+sq=1$.
\end{exer}
\begin{proof}
	Define $T:\P_{n-1}(\C)\times\P_{m-1}(\C)\to\P_{m+n-1}(\C)$ by $T(r,s)=rp+sq$. It can be shown that $T$ is an injective linear map. Because the domain space and target space have the same dimension, $T$ is surjective, completing the proof.
\end{proof}


\section{Eigenvalues and Eigenvectors}
\subsection{Invariant Subspaces}
\begin{thm}
	Suppose $V$ is finite-dimensional, $T\in\L(V)$, and $\lambda\in\F$. Then the following are equivalent.%
	\footnote{The equivalences are useful in that they allow identifying an eigenvalue without explicitly constructing an eigenvector.}
	\begin{enumerate}
		\item $\lambda$ is an eigenvalue of $T$.
		\item $T-\lambda I$ is not injective.
		\item $T-\lambda I$ is not surjective.
		\item $T-\lambda I$ is not invertible.
	\end{enumerate}
\end{thm}

\begin{exer}
	Suppose $T\in\L(V)$ has no eigenvalues and $T^4=I$. Prove that $T^2=-I$.
\end{exer}
\begin{proof}
	$(T^2+I)(T+I)(T-I)=0$. Because $T$ has no eigenvalues, $T+I$ and $T-I$ are both invertible. Hence $T^2+I=0$.
\end{proof}

\begin{exer}
	Suppose that $\lambda_1,\dots,\lambda_n\in\R$ are pairwise distinct. Prove that the list $\e^{\lambda_1x},\dots,\e^{\lambda_nx}$ is linearly independent in $\R^\R$.
\end{exer}
\begin{proof}
	Let $V=\spn(\e^{\lambda_1x},\dots,\e^{\lambda_nx})$.%
	\footnote{Alternatively we can let $V$ be the vector space of differentiable functions on $\R$.}
	Define $D\in\L(V)$ by $Df=f'$. Then $\e^{\lambda x}$ is an eigenvector of $D$ corresponding to $\lambda$. A list of eigenvectors corresponding to distinct eigenvalues is linearly independent.
\end{proof}

\begin{defn}
	Suppose $V$ is finite-dimensional, $T\in\L(V)$, and $U$ is a subspace of $V$ invariant under $T$. The \emph{quotient operator} $T/U\in\L(V/U)$ is defined by
	\[(T/U)(v+U)=Tv+U\]
	for each $v\in V$.
\end{defn}

\begin{exer}
	Suppose $V$ is finite-dimensional, $T\in\L(V)$, and $U$ is a subspace of $V$ invariant under $T$. Prove that each eigenvalue of the quotient operator $T/U$ is an eigenvalue of $T$.
\end{exer}
\begin{proof}
	It suffices to show that $T/U-\lambda I=(T-\lambda I)/U$ is not injective $\implies$ $T-\lambda I$ is not injective. We prove that $T-\lambda I$ is invertible $\implies$ $(T-\lambda I)/U$ is injective.

	Suppose $T-\lambda I$ is invertible. $U$ being invariant under $T$ implies that $U$ is invariant under $T-\lambda I$. Thus $(T-\lambda I)v\in U\iff v\in U$. Suppose $((T-\lambda I)/U)(v+U)=0$. Then $(T-\lambda I)v\in U$, which implies that $v\in U$, i.e., $v+U=0$. That proves the injectivity of $(T-\lambda I)/U$.
\end{proof}

\begin{exer}
	Suppose $V$ is finite-dimensional and $T\in\L(V)$. Prove that $T$ has an eigenvalue if and only if there exists a subspace of $V$ of dimension $\dim V-1$ that is invariant under $T$.%
	\footnote{This proof is inspired by $\M(T)$ and its transpose.}
\end{exer}
\begin{proof}
	We first suppose that $T$ has an eigenvalue $\lambda$. Then there exists $\phi\in V'$ such that $\phi\circ T=T'\phi=\lambda\phi$. Extend $\phi$ to a basis $\phi,\phi_2,\dots,\phi_n$ of $V'$ and let $v,v_2,\dots,v_n$ be a basis of $V$ whose dual basis is $\phi,\phi_2,\dots,\phi_n$. Then $(\phi\circ T)v_k=0$ for every $k$. Because $\phi(T v_k)=0$ for every $k$, $Tv_k\in\spn(v_2,\dots,v_n)$. That proves that $\spn(v_2,\dots,v_n)$ is invariant under $T$.

	To prove the other direction, reverse the steps to obtain an eigenvector of $T'$.
\end{proof}


\subsection{Minimal Polynomials}
\begin{exer}[Companion matrix of a polynomial]
	Suppose $a_0,\dots,a_{n-1}\in\F$. Let $T\in\L(\F^n)$ be such that $\M(T)$ (with respect to the standard basis) is
	\[\begin{bmatrix}
		\hspace{0.3em}0&&&&&-a_0\\
		\hspace{0.3em}1&0&&&&-a_1\\
		\hspace{0.3em}&1&0&&&-a_2\\
		\hspace{0.3em}&&\ddots&&&\vdots\\
		\hspace{0.3em}&&&&0&-a_{n-2}\\
		\hspace{0.3em}&&&&1&-a_{n-1}\\
	\end{bmatrix}\]
	Prove that the minimal polynomial of $T$ is the polynomial%
	\footnote{This exercise implies that every monic polynomial is the minimal polynomial of some operator. Hence an algorithm that could produce exact eigenvalues for each operators on each $\F^n$ does not exist.}
	\[a_0+a_1z+\cdots+a_{n-1}z^{n-1}+z^n.\]
\end{exer}

% \begin{prop}
% 	Suppose $T\in\L(V)$ and $p\in\P(\F)$. Then there exists a unique $r\in\P(\F)$ such that $p(T)=r(T)$ and $\deg r$ is less than the degree of the minimal polynomial of $T$.%
% 	\footnote{This proposition implies that every polynomial applied to an operator can be simplified to a polynomial of smaller degree.}
% \end{prop}

\begin{exer}
	Prove that every operator on a finite-dimensional vector space of dimension at least $2$ has an invariant subspace of dimension $2$.
\end{exer}
\begin{proof}
	Let $T\in\L(V)$ and $\dim V=n$. We use induction on $n$. The base case $n=2$ is trivial. Now suppose $n>2$ and the desired result holds for all smaller positive integers. Let $p$ be the minimal polynomial of $T$.

	If $T$ has an eigenvalue $\lambda$, then $p(z)=q(z)(z-\lambda)$ for some monic polynomial $q$ with $\deg q=\deg p-1$. Because $\rest{q(T)}{\im(T-\lambda I)}=0$, the desired result holds by induction hypothesis if $\dim\im(T-\lambda I)\geq2$. If $T-\lambda I=0$ the desired result trivially holds. If $\dim\im(T-\lambda I)=1$, then $(T-\lambda I)v$ is a scalar multiple of some fixed $u\in V$ for all $v\in V$. Take $w\in V\backslash\spn(u)$ and $\spn(u,w)$ will satisfy the desired property.

	If $T$ has no eigenvalues, then $\F=\R$ and $p(z)=q(z)(z^2+bz+c)$ for some $b,c\in\R$ with $b^2<4c$ and monic polynomial $q$ with $\deg q=\deg p-2$. If $\dim\im(T^2+bT+cI)\geq2$ the desired result holds by induction hypothesis. If $T^2+bT+cI=0$, then any subspace of $V$ with dimension $2$ is a subspace of $\ker(T^2+bT+cI)$, and thus is invariant under $T$. If $\dim\im(T^2+bT+cI)=1$, because $\dim\ker(T^2+bT+cI)$ is even, $n$ is odd and $T$ has an eigenvalue. That completes the proof.
\end{proof}



\end{document}